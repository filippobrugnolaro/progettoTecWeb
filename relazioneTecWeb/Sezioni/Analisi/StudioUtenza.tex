Il progetto \textit{Adrenaline Motocross Park} si propone come piattaforma di prenotazione per un parco divertimenti sportivo dedicato a piloti ed appassionati di motocross. L'intervallo d'età dei piloti si attesta tra i 10 e i 50 anni:
\begin{itemize}
\item Minicross: bambini fino a 14 anni d'eta;
\item Amatori: piloti che non praticano lo sport in maniera non agonistica;
\item Agonisti: piloti che praticano lo sport in maniera agonistica;
\item Elite: piloti professionisti.
\end{itemize} 

I clienti appartenenti a queste categorie costituiscono il tipo utente che maggiormente frequenta il sito web. Questo tipo di utente è già generalmente informato sul mondo del motocross, ne conosce termini tecnici e dinamiche.
Essendo comunque uno sport molto particolare, il linguaggio sarà nella maggior parte semplice, ma potrebbe essere più tecnico in alcuni tratti (descrizione dei tracciati, caratteristiche dei mezzi a noleggio). Si prevede che questo tipo di utente sarà quello che maggiormente utilizzerà la parte riservata del sito,  al fine di gestire le sue prenotazioni.

L'altra grande categoria di utenti è rappresentata dall'utente generico, ovvero colui che non conosce in modo approfondito questo sport. Generalmente si può riconoscere questo tipo di utenza nelle persone che vengono ad assistere agli allenamenti dei piloti o più in generale in qualsiasi simpatizzante dello sport. Si prevede che questo tipo di utente sarà quello che farà più uso della parte pubblica del sito, cercando date di apertura o altre informazioni non collegate con il sistema di prenotazioni.

In generale il sito deve essere adeguatamente specifico nel fornire informazioni tecniche utili al frequentatore, ma deve essere anche sufficientemente generico per poter permettere anche a chi non è dell'ambiente di informarsi (e magari diventare frequentatore).