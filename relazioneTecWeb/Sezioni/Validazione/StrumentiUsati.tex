Per la validazione dei file HTML è stato usato il validatore di W3C\footnote{\href{https://validator.w3.org/}{W3C Validator}}.\\
Per il CSS del sito è stato usato il validatore CSS\footnote{\href{http://www.css-validator.org/}{CSS Validator}} sempre offerto dal W3C.\\
L'accessibilità, a livello di test automatici, è stata controllata tramite lo strumento WAVE Evaluation\footnote{\href{https://wave.webaim.org/}{Wave Evaluation}}.\\
Infine, per la validazione sintattica degli script PHP ed il JavaScript, sono stati usati PhpCodeChecker\footnote{\href{https://phpcodechecker.com/}{PHPCodeChecker}} ed Esprima\footnote{\href{https://esprima.org/demo/validate.html}{Esprima}}.