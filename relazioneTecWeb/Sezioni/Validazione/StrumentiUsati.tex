Per la validazione dei file HTML è stato usato il validatore di W3C\footnote{https://validator.w3.org/}.
Per il CSS del sito è stato usato il validatore CSS\footnote{http://www.css-validator.org/} sempre offerto dal W3C.
L'accessibilità è stata controllata tramite lo strumento WAVE Evaluation\footnote{https://wave.webaim.org/}.
Infine, per il PHP ed il Javascript, anche se non richiesto, abbiamo utilizzato rispettivamente PhpCodeChecker\footnote{https://phpcodechecker.com/} ed Esprima\footnote{https://esprima.org/demo/validate.html}.