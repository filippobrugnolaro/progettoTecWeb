Lo strumento di validazione di W3C ci ha consentito di validare anche l’HTML prodotto dalle pagine PHP, poichè permette di incollare direttamente il codice HTML prodotto dallo script. È il motivo per cui è stato utilizzato per validare tutto il codice HTML. 

Non è stato possibile testare tutte le possibili combinazioni degli esiti degli script PHP (richiedeva la presenza di situazioni complesse da simulare). Per questo motivo, oltre al tool indicato, è stata effettuata la verifica manuale in stile \textit{walkthrough} degli script PHP, controllando che tutti i testi aggiunti dinamicamente venissero inclusi nei corretti tag.