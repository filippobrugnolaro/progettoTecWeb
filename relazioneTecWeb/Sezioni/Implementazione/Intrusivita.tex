Di seguito verrà descritto il livello di intrusività dei linguaggi rispetto al codice sorgente HTML:
\begin{itemize}
    \item \textbf{CSS}: \underline{non} intrusivo. Tutto il codice CSS è stato inserito in un file a parte, evitando del tutto pratiche come il
    CSS inline o embedded.
    \item \textbf{PHP}: \underline{non} intrusivo. Sebbene ci sia del codice HTML all'interno di alcuni file PHP, la corrispondente
    pagina HTML non contiene codice PHP. La motivazione per la quale c'è del codice HTML in file PHP è che spesso i dati, essendo molto dinamici,
    cambiano spesso in base sia all'input dell'utente che dell'amministratore. Dunque in base alla presenza o meno di alcuni di questi 
    (per esempio le date d'ingresso) possono comparire o meno determinati elementi HTML.
    \item \textbf{Javascript}: \underline{non} intrusivo. Tutto il codice javascript è stato inserito in un file a parte.
\end{itemize}