Sql è stato utilizzato per la codifica del database. Il database è composto dalle seguenti tabelle:
\begin{itemize}
    \item \textbf{data\_disponibile}: contiene le \textit{date} di apertura dell'impianto che vengono
    inserite dall'amministratore; viene inoltre stabilito il numero di \textit{posti} (esclusi i piloti
    partecipanti ai corsi) disponibili in ogni data di apertura.
    \item \textbf{ingressi\_entrata}: contiene la \textit{data} e la licenza dell'\textit{utente} che ha prenotato l'ingresso
    \item \textbf{ingressi\_lezione}: contiene la \textit{data} e la licenza dell'\textit{utente} che ha prenotato la lezione
    \item \textbf{lezione}: contiene tutte le lezioni inserite dall'amministratore; ogni lezione contiene la \textit{data} in cui verrà svolta, l'\textit{istruttore}
    che la svolgerà in una \textit{pista} assegnata, la descrizione e il numero di posti disponibili.
    \item \textbf{messaggio}: contiene
    \item \textbf{moto}:
    \item \textbf{noleggio}:
    \item \textbf{pista}:
    \item \textbf{utente}:
\end{itemize}