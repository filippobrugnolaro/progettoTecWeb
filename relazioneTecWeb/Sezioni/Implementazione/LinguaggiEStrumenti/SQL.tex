\textit{SQL} è stato utilizzato per la codifica del database. Il database è composto dalle seguenti tabelle:
\begin{itemize}
    \item \textbf{\textit{data\_disponibile}}: contiene le \textit{date} di apertura dell'impianto che vengono
    inserite dall'amministratore; viene inoltre stabilito il numero di \textit{posti} (esclusi i piloti
    partecipanti ai corsi) disponibili in ogni data di apertura;
    
    \item \textbf{\textit{ingressi\_entrata}}: contiene la \textit{data} e il codice fiscale dell'\textit{utente} che ha prenotato l'ingresso;
    
    \item \textbf{\textit{ingressi\_lezione}}: contiene l'id del \textit{corso} e il codice fiscale dell'\textit{utente} che ha prenotato il corso;
    
    \item \textbf{\textit{lezione}}: contiene tutti i corsi inseriti dall'amministratore; ogni corso contiene la \textit{data}
    in cui verrà svolto, l'\textit{istruttore} che lo svolgerà in una \textit{pista}, la \textit{descrizione} e il numero di \textit{posti} disponibili;
    
    \item \textbf{\textit{messaggio}}: contiene l'\textit{oggetto}, il \textit{testo} e la \textit{data} di invio del messaggio; per identificare
    chi lo ha inviato si hanno \textit{nominativo}, \textit{email} e \textit{telefono};
    
    \item \textbf{\textit{moto}}: contiene le moto disponibili dell'impianto; per identificarle si ha la \textit{marca}, il \textit{modello},
    la \textit{cilindrata}, l'\textit{anno} di produzione e un identificativo progressivo;
    
    \item \textbf{\textit{noleggio}}: contiene informazioni riguardo il noleggio in una determinata \textit{data} di \textit{attrezzatura} o
    \textit{moto} da parte di un \textit{utente};
    
    \item \textbf{\textit{pista}}: contiene tutte le piste inserite dall'amministratore; ogni pista contiene la \textit{lunghezza}, il tipo di \textit{terreno}, la \textit{descrizione} e gli orari di \textit{apertura} e \textit{chiusura}; è possibile inserire una \textit{foto} del tracciato stesso;
    
    \item \textbf{\textit{utente}}: contiene tutti le informazioni degli utenti/amministratori (in base al loro \textit{ruolo} all'interno del sito);
    è dunque presente \textit{cognome}, \textit{nome} della persona con relativa data di \textit{nascita}, codice fiscale (\textit{cf}) e
    numero di \textit{telefono}; vengono memorizzate anche le credenziali come l'\textit{email}, l'\textit{username} e la \textit{password}.
\end{itemize}
\begin{figure}[H]
    \includegraphics[scale=0.4]{./res/schemaSQL.pdf}
    \centering
    \caption{Schema SQL}
\end{figure}
