Il gruppo ha utilizzato il linguaggio HTML5.\footnote{https://html.spec.whatwg.org/multipage/}
Per assicurare un codice corretto, sono state seguite le linee guida del corso di Tecnologie Web.
Il codice è stato validato utilizzando il tool di validazione W3C\footnote{https://validator.w3.org/}.\\
In particolari si ha riposto attenzione alle seguenti peculiarità:
\begin{itemize}
    \item \textbf{Chiusura tag:} ogni tag deve essere chiuso(<tag></tag> oppure <tag/>);
    \item \textbf{Metatag:} nella sezione header, devono essere inseriti i metatag necessari per migliorare l'accessibilità verso i motori
    di ricerca. Questo permette al sito di avere una migliore visibilità in internet. (sezione relativa alle Possibili ricerche sui motori di ricerca);
    \item \textbf{Separazione struttura-presentazione-comportamento:} il codice HTML non deve contenere CSS o script. Questi devono essere
    scritti in file separati e importati nell'header.
\end{itemize}

