\externaldocument[R-]{relazione}

Il gruppo ha utilizzato il linguaggio HTML5\footnote{https://html.spec.whatwg.org/multipage/} cercando mi mantenere il più possibile la
retrocompatibilità con XHTML\footnote{http://www.w3.org/TR/2018/SPSD-xhtml1-20180327/}, così da rendere il sito accessibile anche
su dispositivi con browser più obsoleti. Inoltre la verbosità del secondo rispetto al primo ha contribuito notevolmente sulla scelta finale.
Per assicurare un codice corretto, sono state seguite le linee guida del corso di Tecnologie Web.
Il codice è stato validato utilizzando il tool di validazione W3C\footnote{https://validator.w3.org/}.\\
In particolari si ha riposto attenzione alle seguenti peculiarità:
\begin{itemize}
    \item \textbf{Chiusura tag:} ogni tag deve essere chiuso(<tag></tag> oppure <tag/>);
    \item \textbf{Metatag:} nella sezione header, devono essere inseriti i metatag necessari per migliorare l'accessibilità verso i motori
    di ricerca. Questo permette al sito di avere una migliore visibilità in internet. (sezione relativa alle Possibili ricerche sui motori di ricerca);
    \item \textbf{Separazione struttura-presentazione-comportamento:} il codice HTML non deve contenere CSS o script. Questi devono essere
    scritti in file separati e importati nell'header;
    \item \textbf{Struttura:} il codice HTML non deve sostituire il CSS.
\end{itemize}

