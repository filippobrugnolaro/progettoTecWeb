Il gruppo ha utilizzato il linguaggio di markup \textit{HTML5}\footnote{https://html.spec.whatwg.org/multipage/} per la modellazione delle pagine web.
Per assicurare un codice corretto, sono state seguite le linee guida del corso di Tecnologie Web. Il codice è stato validato utilizzando il tool di validazione messo a disposizione da \textit{W3C}\footnote{https://validator.w3.org/}.\\
In particolare l'attenzione si è focalizzata sulle seguenti peculiarità:
\begin{itemize}
    \item \textbf{Meta-tag}: nella sezione head dei documenti, devono essere inseriti i meta-tag necessari per migliorare l'accessibilità verso i motori di ricerca. Questo permette al sito di avere una migliore visibilità nelle \textit{SERP}. In particolare si richiede anche di istruire i bot sui contenuti da indicizzare o meno.
    \item \textbf{Separazione struttura/contenuto - presentazione - comportamento}: il codice HTML non deve contenere codice CSS embedded, stili inline o codice di scripting (PHP o JavaScript). Questi devono essere codificati in file separati e importati nella sezione head dei documenti \textit{HTML} che li richiedono.
\end{itemize}

