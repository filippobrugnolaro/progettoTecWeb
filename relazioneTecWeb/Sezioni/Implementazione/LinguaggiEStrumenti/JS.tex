Il linguaggio Javascript è stato utilizzato per due scopi principalmente:
\begin{enumerate}
    \item per convalidare i dati inseriti durante i form sia della parte pubblica che della parte privata sia dell'utente che dell'amministratore;
    \item per aggiornare costantemente il tipo di scelte che un utente può fare in base alle scelte effettuate dagli altri utenti.
\end{enumerate}
Avendo suddiviso il progetto in cartelle, per ogni pagina che necessitasse di una validazione di un form abbiamo creato il suo corrispondente file
javascript (nomeValidation.js) contente i controlli utili per la validazione; per ogni pagina che invece necessitasse un aggiornamento continuo dei dati (vedi i corsi e gli
ingressi nell'area utente), è stato creato il suo corrispondente file javascript (nome.js) contenente la chiamata AJAX al server.