Per modellare il layout di presentazione del sito è stato utilizzato il linguaggio di formattazione \textit{CSS}. Il contenuto della presentazione viene diviso in 3 sezioni e 2 file.

Il primo file contiene tutte le clausole CSS relative alla formattazione dei documenti che verranno visualizzati dall'utente nel sito web. Si è utilizzata la tecnica di sviluppo \textit{"Mobile First"}, in modo da individuare le informazioni di primaria importanza da visualizzare. Attraverso l'utilizzo delle \textit{media query} si è poi adattato il layout per schermi più grandi di 600px.

Il secondo file contiene le clausole CSS per la formattazione dei documenti nel caso di stampa. Si è voluta implementare anche questo layout in quanto ritenuto utile per l'utente nel momento in cui volesse conservare le prenotazioni anche in formato cartaceo.

In entrambi i file è stato utilizzato la versione 3 del linguaggio. Sono infatti presenti varie regole disponibili con CSS3 (flexbox, variabili, selettori, ...). Questa scelta è stata dettata da due fattori principali:
\begin{itemize}
\item L'utilizzo principale del sito avverrà attraverso smartphone;
\item La fascia di età target è relativamente giovane.
\end{itemize}

Questi due fattori hanno portato il gruppo a concludere che nelle stragrande maggioranza dei casi, l'utente avrà a disposizione un cellulare moderno ed aggiornato, rendendo perfettamente compatibile l'utilizzo di queste particolari funzionalità presenti, per ora, solo nei browser moderni.