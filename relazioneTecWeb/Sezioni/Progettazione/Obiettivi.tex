Nello sviluppo del sito il gruppo si è imposto alcuni intransigenti obiettivi:
\begin{itemize}
    \item \textbf{Separazione contenuto/struttura - presentazione - comportamento}: Il più importante, in quanto il suo raggiungimento comporta:
	\begin{itemize}
	\item Maggior facilità in una eventuale futura manutenzione o espansione;
	\item Minor peso complessivo del sito e conseguente miglior posizionamento nelle ricerche;
	\end{itemize}	    

    Si sono dunque definite le macro-componenti in diverse regioni:
    \begin{itemize}
    	\item Struttura e contenuto: informazioni e dati in documenti HTML;
    	\item Presentazione: figli di stile CSS;
    	\item Comportamento:
    		\begin{itemize}
    		\item Lato server: script PHP;
    		\item Lato client: script JavaScript.
    		\end{itemize}
	\end{itemize}    
	
	La completa separazione delle regioni non è sempre stata possibile, per ragioni derivanti dal dominio d'interesse e per agevolare l'esperienza dell'utente.	
	
   \item \textbf{Accessibilità}: Il sito deve poter essere fruibile agevolmente dal maggior numero utenti possibile, compresi quelli con gravi disabilità visive e/o
    motorie. Alcune contromisure significative che sono state adottate:
    \begin{itemize}
        \item Lettura corretta delle tabelle da parte di dispositivi di assistenza (screen reader, ...);
        \item Testo alternativo per le immagini di contenuto;
        \item Testi e link con buoni livelli di contrasto;
    \end{itemize}
    
    \textbf{NB}: icone e immagini di presentazione, poiché decorative, non hanno alcun testo alternativo dato che, se venissero rimosse, l'utente capirebbe lo stesso ciò di cui si sta parlando.
    
L'accessibilità deve essere garantita indipendentemente dal tipo di dispositivo. Deve essere preferito lo sviluppo \textit{"Mobile First"}, data l'utenza relativamente giovane e il tipo di operazioni da eseguire (prenotazioni). È essenziale garantire all'utente la possibilità di svolgere tutte le operazioni da smartphone, al fine di rendere l'esperienza utente il più agevole possibile.
    
    \item \textbf{Flessibilità}: Il sito deve essere consultabile da varie tipologie di dispositivi, smartphone compresi. Deve, inoltre, essere adattabile a differenti dimensioni di schermo con il minor sforzo possibile.
\end{itemize}
