Nello sviluppo del sito il gruppo si è imposto alcuni intransigenti obiettivi:
\begin{itemize}
    \item \textbf{Separazione struttura-presentazione-comportamento}: Il più importante, in quanto il suo raggiungimento
    comporta una maggior facilità in una eventuale futura manutenzione o espansione.
    Si sono dunque definite le componenti di stile nei fogli CSS, il contenuto nelle pagine PHP e HTML e il comportamento nei file javascript.
    \item \textbf{Accessibilità}: Il sito deve poter essere fruibile agevolmente dal maggior
    numero utenti possibile, compresi quelli con gravi disabilità visive e/o
    motorie. Alcuni contromisure significative che sono state adottate abbiamo:
    \begin{itemize}
        \item[$\circ$] lettura corretta delle tabelle;
        \item[$\circ$] testo alternativo per le immagini;
        \item[$\circ$] testi e link con buoni livelli di contrasto;
    \end{itemize}
    \textbf{NB}: le icone, poichè decorative, non hanno alcun testo alternativo dato che, se venissero rimosse, l'utente
    capirebbe lo stesso ciò di cui si sta parlando.
    \item \textbf{Flessibilità}: Il sito deve essere consultabile da varie tipologie di dispositivi,
    smartphone compresi. Deve, inoltre, essere adattabile a differenti dimensioni
    di schermo con il minor sforzo possibile.
\end{itemize}
