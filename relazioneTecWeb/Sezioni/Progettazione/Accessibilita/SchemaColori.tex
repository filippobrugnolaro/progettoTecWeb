La scelta dei colori ha un impatto fondamentale per quanto riguarda l'accessibilità del sito, per questo è di primaria importanza. Innanzitutto vi è un contrasto elevato tra le componenti principali del sito (sistema di navigazione, contenuto principale e form), per facilitare la lettura del contenuto anche a chi soffre di disturbi visivi.

Si è deciso di rompere la convenzione esterna che vuole i link non visitati blu e i visitati viola, questo per mantenere coerenza con i colori di bandiera dell'impianto (bianco e verde scuro). Il colore dei link è stato cambiato a grigio chiaro per i link non visitati e arancione per quelli visitati. Gli unici link 'insensibili' a questa scelta sono quelli che rimandano alla cima della pagina, in quanto non considerati dei veri link.

Le convenzioni interne vengono assolutamente rispettate in ogni componente ed elemento del sito.

Nella barra di navigazione i link vengono evidenziati al passaggio del puntatore, diventando bianchi. All'interno di una pagina, la particolare voce del menu non sarà selezionabile e diventa un semplice contenuto testuale con carattere ingrandito.

Sono stati utilizzati principalmente due colori di sfondo, utili a separare aree funzionali diverse all'interno del sito. Mentre lo sfondo del sito ha un colore simile ad un arancione sbiadito, ogni componente che sta nel pannello di contenuto (form, article, ...) viene evidenziata con un arancione più accesso.